\subsection{Universidad de Sudáfrica}

Resultados obtenidos en el monitoreo:\\

\smallskip
\begin{tabular}{| l | c | c | c | c |}
\hline
Hop & IP &  RTT promedio (s)  & deltaRTT promedio & Ubicacion\\ 
\hline
1 & 192.168.11.1 & 0.00773384835985 & 0.00773384835985 & Argentina, Buenos Aires\\
\hline
2 & 10.21.128.1 & 0.12322970799 & 0.115495859631 & Argentina, Buenos Aires\\
\hline
3 & 10.242.0.201 & 0.0379023551941 & 0 & Argentina, Buenos Aires\\
\hline
4 & 200.63.150.242 & 0.16964647505 & 0.131744119856 & Argentina\\
\hline
5 & 200.63.150.241 & 0.0638248622417 & 0 & Argentina\\
\hline
6 & 200.51.208.62 & 0.0236063798269 & 0 & Argentina, Buenos Aires\\
\hline
7 & 213.140.39.118 & 0.124509599474 & 0.100903219647 & Spain\\
\hline
8 & 5.53.5.62 & 0.208739678065 & 0.0842300785912 & Spain\\
\hline
9 & 94.142.125.165 & 0.151147206624 & 0 & Spain\\
\hline
10 & 4.69.158.245 & 0.508760134379 & 0.357612927755 & United States\\
\hline
11 & 4.69.158.245 & 0.43449666765 & 0 & United States\\
\hline
12 & 213.242.110.114 & 0.528451919556 & 0.0939552519057 & Ireland, Boyle\\
\hline
13 & 80.64.96.228 & 0.536008971078 & 0.00755705152239 & Russia, Redkino\\
\hline
14 & 80.64.103.9 & 0.433499839571 & 0 & Russia, Saint Petersburg\\
\hline
15 & 185.44.12.155 & 0.452484236823 & 0.0189843972524 & Russia\\
\hline
16 & 185.44.15.196 & 0.337167978287 & 0 & Russia\\
\hline
\end{tabular}
\bigskip

\textbf{Paquetes enviados: 261 / Paquetes no respondidos: 70}\\

\textbf{Seis outliers, hops: 6, 7, 11, 13, 17 y 19}\\

Los primeros saltos que realiza son similares (sino idénticos) a los que se obtuvieron en la prueba realizada hacia 
Pekín, no parece coincidir la información de geolocalización con los datos obtenidos, los RTTs hasta el hop 5 difieren en
muy poco tiempo, lo que nos hace pensar que salto de Argentina hacia Italia no es correcto.
Sí parece producirse un salto en el hop 6, detectado como outlier. Sucede lo mismo con el salto 17 de EEUU a Sudáfrica.

En total se detectaron 6 outliers y 2 posibles saltos intercontinentales.

A continuación mostramos un gráfico con los RTT entre saltos y otro con los ZRTT\footnote{ZRTT = $(X_i - \bar{X}) / S$}  entre saltos. También así el planisferio con los saltos graficados.

\includegraphics[scale=0.65]{imagenes/sudafrica/RTTs.png} 

\includegraphics[scale=0.65]{imagenes/sudafrica/ZRTTs.png}

\begin{center}
\includegraphics[scale=0.8]{imagenes/sudafrica/sudafrica.pdf} 
\end{center}