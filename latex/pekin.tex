\subsection{Universidad de Pekín}

Resultados obtenidos en el monitoreo:\\

\smallskip
\begin{tabular}{| l | c | c | c | c |}
\hline
Hop & IP &  RTT promedio (s)  & deltaRTT promedio & Ubicacion\\ 
\hline
1 & 192.168.11.1 & 0.00773384835985 & 0.00773384835985 & Argentina, Buenos Aires\\
\hline
2 & 10.21.128.1 & 0.12322970799 & 0.115495859631 & Argentina, Buenos Aires\\
\hline
3 & 10.242.0.201 & 0.0379023551941 & 0 & Argentina, Buenos Aires\\
\hline
4 & 200.63.150.242 & 0.16964647505 & 0.131744119856 & Argentina\\
\hline
5 & 200.63.150.241 & 0.0638248622417 & 0 & Argentina\\
\hline
6 & 200.51.208.62 & 0.0236063798269 & 0 & Argentina, Buenos Aires\\
\hline
7 & 213.140.39.118 & 0.124509599474 & 0.100903219647 & Spain\\
\hline
8 & 5.53.5.62 & 0.208739678065 & 0.0842300785912 & Spain\\
\hline
9 & 94.142.125.165 & 0.151147206624 & 0 & Spain\\
\hline
10 & 4.69.158.245 & 0.508760134379 & 0.357612927755 & United States\\
\hline
11 & 4.69.158.245 & 0.43449666765 & 0 & United States\\
\hline
12 & 213.242.110.114 & 0.528451919556 & 0.0939552519057 & Ireland, Boyle\\
\hline
13 & 80.64.96.228 & 0.536008971078 & 0.00755705152239 & Russia, Redkino\\
\hline
14 & 80.64.103.9 & 0.433499839571 & 0 & Russia, Saint Petersburg\\
\hline
15 & 185.44.12.155 & 0.452484236823 & 0.0189843972524 & Russia\\
\hline
16 & 185.44.15.196 & 0.337167978287 & 0 & Russia\\
\hline
\end{tabular}
\bigskip

\textbf{Paquetes enviados: 261 / Paquetes no respondidos: 48}\\

\textbf{Ocho outliers, hops: 5, 8, 11, 12, 15, 16, 17 y 24}\\

[MEJORAR ESTE ANALISIS]
fruta en el salto 4, el salto 5 puede ser italia? puede ser, es un outlier fijarse
salto a eeuu en el 8, notorio, de hecho tmb es outlier
11 y 12 outliers no se que carajo, 15 tmp
16 salto intercont. a china. ya llegamos a la china? outlier cimbala tenía razón
17 outlier pero seguimos en china
24 outlier, decir algo como que puede ser que el firewall del tibet nos chequee el paquete (?)

A continuación mostramos un gráfico con los RTT entre saltos y otro con los ZRTT\footnote{ZRTT = $(X_i - \bar{X}) / S$}  entre saltos. También así el planisferio con los saltos graficados.

\includegraphics[scale=0.65]{imagenes/pekin/RTTs.png} 

\includegraphics[scale=0.65]{imagenes/pekin/ZRTTs.png} 

\begin{center}
\includegraphics[scale=0.8]{imagenes/pekin/pekin.pdf} 
\end{center}
