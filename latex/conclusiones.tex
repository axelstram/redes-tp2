\section{Conclusiones}

En todos los casos de monitoreo observamos la anomalía típica de los \textit{Missing Hops}. Esto probablemente se deba a que ciertas redes están protegidas por firewalls. Comparando nuestros resultados con los de la herramienta \textit{traceroute}, vimos que ésta se encontraba ante los mismos problemas. También, de manera no sorprendente, nos encontramos con la anomalía \textit{Missing Destination}, muy similar a la anterior. Para esto nos basamos en que por ejemplo, en el caso de Helsinki, los últimos hops que responden se encuentran en Finlandia, seguidos de una serie de hops que no responden. Dado que la mayoría de las Universidades cuentan con firewalls, asumimos que el paquete llegó a destino pero los firewalls no permiten que se respondan los paquetes ICMP.

También creemos habernos encontrado con anomalías del tipo \textit{False Rountrip Time}. Esta anomalía es más difícil de aseverar, pero por los picos encontrados en los gráficos de $\Delta RTT$ promedio, creemos que estamos ante un caso de esta anomalía, lo cual en algunas ocasiones pudimos verificar también con \textit{traceroute}.

Otra anomalía que encontramos fue la de \textit{Loops and Circles}, por ejemplo en el caso de la Universidad de Pekín, con los hops 6 y 7; la de San Petersburgo, en los hops 10 y 11; y en la Universidad de Sudáfrica, en los hops 6 y 7 (que parecerían estar en la misma red que los hops 6 y 7 de la Universidad de Pekín).

Creemos que no es posible mejorar demasiado la estimación de outliers del método Cimbala con un valor fijo, ya que con los valores de la tabla observamos una amplia variedad en la cantidad de outliers detectados. En algunos casos, como el de la Universidad de Helsinki, consideramos que los outliers detectados son de interés, mientras que en otros casos, como el de la Universidad de Pekín se detectan demasiados outliers, muchos de los cuáles creemos que son falsos positivos. No creemos que con un valor fijo se pueda mejorar demasiado esto; podrá disminuir la cantidad de outliers en la Universidad de Pekín, probablemente a costa de perder outliers de interés en la Universidad de Helsinki.